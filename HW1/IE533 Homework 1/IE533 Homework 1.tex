\documentclass[11pt]{article}

    \usepackage[breakable]{tcolorbox}
    \usepackage{parskip} % Stop auto-indenting (to mimic markdown behaviour)
    

    % Basic figure setup, for now with no caption control since it's done
    % automatically by Pandoc (which extracts ![](path) syntax from Markdown).
    \usepackage{graphicx}
    % Maintain compatibility with old templates. Remove in nbconvert 6.0
    \let\Oldincludegraphics\includegraphics
    % Ensure that by default, figures have no caption (until we provide a
    % proper Figure object with a Caption API and a way to capture that
    % in the conversion process - todo).
    \usepackage{caption}
    \DeclareCaptionFormat{nocaption}{}
    \captionsetup{format=nocaption,aboveskip=0pt,belowskip=0pt}

    \usepackage{float}
    \floatplacement{figure}{H} % forces figures to be placed at the correct location
    \usepackage{xcolor} % Allow colors to be defined
    \usepackage{enumerate} % Needed for markdown enumerations to work
    \usepackage{geometry} % Used to adjust the document margins
    \usepackage{amsmath} % Equations
    \usepackage{amssymb} % Equations
    \usepackage{textcomp} % defines textquotesingle
    % Hack from http://tex.stackexchange.com/a/47451/13684:
    \AtBeginDocument{%
        \def\PYZsq{\textquotesingle}% Upright quotes in Pygmentized code
    }
    \usepackage{upquote} % Upright quotes for verbatim code
    \usepackage{eurosym} % defines \euro

    \usepackage{iftex}
    \ifPDFTeX
        \usepackage[T1]{fontenc}
        \IfFileExists{alphabeta.sty}{
              \usepackage{alphabeta}
          }{
              \usepackage[mathletters]{ucs}
              \usepackage[utf8x]{inputenc}
          }
    \else
        \usepackage{fontspec}
        \usepackage{unicode-math}
    \fi

    \usepackage{fancyvrb} % verbatim replacement that allows latex
    \usepackage{grffile} % extends the file name processing of package graphics
                         % to support a larger range
    \makeatletter % fix for old versions of grffile with XeLaTeX
    \@ifpackagelater{grffile}{2019/11/01}
    {
      % Do nothing on new versions
    }
    {
      \def\Gread@@xetex#1{%
        \IfFileExists{"\Gin@base".bb}%
        {\Gread@eps{\Gin@base.bb}}%
        {\Gread@@xetex@aux#1}%
      }
    }
    \makeatother
    \usepackage[Export]{adjustbox} % Used to constrain images to a maximum size
    \adjustboxset{max size={0.9\linewidth}{0.9\paperheight}}

    % The hyperref package gives us a pdf with properly built
    % internal navigation ('pdf bookmarks' for the table of contents,
    % internal cross-reference links, web links for URLs, etc.)
    \usepackage{hyperref}
    % The default LaTeX title has an obnoxious amount of whitespace. By default,
    % titling removes some of it. It also provides customization options.
    \usepackage{titling}
    \usepackage{longtable} % longtable support required by pandoc >1.10
    \usepackage{booktabs}  % table support for pandoc > 1.12.2
    \usepackage{array}     % table support for pandoc >= 2.11.3
    \usepackage{calc}      % table minipage width calculation for pandoc >= 2.11.1
    \usepackage[inline]{enumitem} % IRkernel/repr support (it uses the enumerate* environment)
    \usepackage[normalem]{ulem} % ulem is needed to support strikethroughs (\sout)
                                % normalem makes italics be italics, not underlines
    \usepackage{mathrsfs}
    

    
    % Colors for the hyperref package
    \definecolor{urlcolor}{rgb}{0,.145,.698}
    \definecolor{linkcolor}{rgb}{.71,0.21,0.01}
    \definecolor{citecolor}{rgb}{.12,.54,.11}

    % ANSI colors
    \definecolor{ansi-black}{HTML}{3E424D}
    \definecolor{ansi-black-intense}{HTML}{282C36}
    \definecolor{ansi-red}{HTML}{E75C58}
    \definecolor{ansi-red-intense}{HTML}{B22B31}
    \definecolor{ansi-green}{HTML}{00A250}
    \definecolor{ansi-green-intense}{HTML}{007427}
    \definecolor{ansi-yellow}{HTML}{DDB62B}
    \definecolor{ansi-yellow-intense}{HTML}{B27D12}
    \definecolor{ansi-blue}{HTML}{208FFB}
    \definecolor{ansi-blue-intense}{HTML}{0065CA}
    \definecolor{ansi-magenta}{HTML}{D160C4}
    \definecolor{ansi-magenta-intense}{HTML}{A03196}
    \definecolor{ansi-cyan}{HTML}{60C6C8}
    \definecolor{ansi-cyan-intense}{HTML}{258F8F}
    \definecolor{ansi-white}{HTML}{C5C1B4}
    \definecolor{ansi-white-intense}{HTML}{A1A6B2}
    \definecolor{ansi-default-inverse-fg}{HTML}{FFFFFF}
    \definecolor{ansi-default-inverse-bg}{HTML}{000000}

    % common color for the border for error outputs.
    \definecolor{outerrorbackground}{HTML}{FFDFDF}

    % commands and environments needed by pandoc snippets
    % extracted from the output of `pandoc -s`
    \providecommand{\tightlist}{%
      \setlength{\itemsep}{0pt}\setlength{\parskip}{0pt}}
    \DefineVerbatimEnvironment{Highlighting}{Verbatim}{commandchars=\\\{\}}
    % Add ',fontsize=\small' for more characters per line
    \newenvironment{Shaded}{}{}
    \newcommand{\KeywordTok}[1]{\textcolor[rgb]{0.00,0.44,0.13}{\textbf{{#1}}}}
    \newcommand{\DataTypeTok}[1]{\textcolor[rgb]{0.56,0.13,0.00}{{#1}}}
    \newcommand{\DecValTok}[1]{\textcolor[rgb]{0.25,0.63,0.44}{{#1}}}
    \newcommand{\BaseNTok}[1]{\textcolor[rgb]{0.25,0.63,0.44}{{#1}}}
    \newcommand{\FloatTok}[1]{\textcolor[rgb]{0.25,0.63,0.44}{{#1}}}
    \newcommand{\CharTok}[1]{\textcolor[rgb]{0.25,0.44,0.63}{{#1}}}
    \newcommand{\StringTok}[1]{\textcolor[rgb]{0.25,0.44,0.63}{{#1}}}
    \newcommand{\CommentTok}[1]{\textcolor[rgb]{0.38,0.63,0.69}{\textit{{#1}}}}
    \newcommand{\OtherTok}[1]{\textcolor[rgb]{0.00,0.44,0.13}{{#1}}}
    \newcommand{\AlertTok}[1]{\textcolor[rgb]{1.00,0.00,0.00}{\textbf{{#1}}}}
    \newcommand{\FunctionTok}[1]{\textcolor[rgb]{0.02,0.16,0.49}{{#1}}}
    \newcommand{\RegionMarkerTok}[1]{{#1}}
    \newcommand{\ErrorTok}[1]{\textcolor[rgb]{1.00,0.00,0.00}{\textbf{{#1}}}}
    \newcommand{\NormalTok}[1]{{#1}}

    % Additional commands for more recent versions of Pandoc
    \newcommand{\ConstantTok}[1]{\textcolor[rgb]{0.53,0.00,0.00}{{#1}}}
    \newcommand{\SpecialCharTok}[1]{\textcolor[rgb]{0.25,0.44,0.63}{{#1}}}
    \newcommand{\VerbatimStringTok}[1]{\textcolor[rgb]{0.25,0.44,0.63}{{#1}}}
    \newcommand{\SpecialStringTok}[1]{\textcolor[rgb]{0.73,0.40,0.53}{{#1}}}
    \newcommand{\ImportTok}[1]{{#1}}
    \newcommand{\DocumentationTok}[1]{\textcolor[rgb]{0.73,0.13,0.13}{\textit{{#1}}}}
    \newcommand{\AnnotationTok}[1]{\textcolor[rgb]{0.38,0.63,0.69}{\textbf{\textit{{#1}}}}}
    \newcommand{\CommentVarTok}[1]{\textcolor[rgb]{0.38,0.63,0.69}{\textbf{\textit{{#1}}}}}
    \newcommand{\VariableTok}[1]{\textcolor[rgb]{0.10,0.09,0.49}{{#1}}}
    \newcommand{\ControlFlowTok}[1]{\textcolor[rgb]{0.00,0.44,0.13}{\textbf{{#1}}}}
    \newcommand{\OperatorTok}[1]{\textcolor[rgb]{0.40,0.40,0.40}{{#1}}}
    \newcommand{\BuiltInTok}[1]{{#1}}
    \newcommand{\ExtensionTok}[1]{{#1}}
    \newcommand{\PreprocessorTok}[1]{\textcolor[rgb]{0.74,0.48,0.00}{{#1}}}
    \newcommand{\AttributeTok}[1]{\textcolor[rgb]{0.49,0.56,0.16}{{#1}}}
    \newcommand{\InformationTok}[1]{\textcolor[rgb]{0.38,0.63,0.69}{\textbf{\textit{{#1}}}}}
    \newcommand{\WarningTok}[1]{\textcolor[rgb]{0.38,0.63,0.69}{\textbf{\textit{{#1}}}}}


    % Define a nice break command that doesn't care if a line doesn't already
    % exist.
    \def\br{\hspace*{\fill} \\* }
    % Math Jax compatibility definitions
    \def\gt{>}
    \def\lt{<}
    \let\Oldtex\TeX
    \let\Oldlatex\LaTeX
    \renewcommand{\TeX}{\textrm{\Oldtex}}
    \renewcommand{\LaTeX}{\textrm{\Oldlatex}}
    % Document parameters
    % Document title
    \title{IE533 Homework 1}
    
    
    
    
    
% Pygments definitions
\makeatletter
\def\PY@reset{\let\PY@it=\relax \let\PY@bf=\relax%
    \let\PY@ul=\relax \let\PY@tc=\relax%
    \let\PY@bc=\relax \let\PY@ff=\relax}
\def\PY@tok#1{\csname PY@tok@#1\endcsname}
\def\PY@toks#1+{\ifx\relax#1\empty\else%
    \PY@tok{#1}\expandafter\PY@toks\fi}
\def\PY@do#1{\PY@bc{\PY@tc{\PY@ul{%
    \PY@it{\PY@bf{\PY@ff{#1}}}}}}}
\def\PY#1#2{\PY@reset\PY@toks#1+\relax+\PY@do{#2}}

\@namedef{PY@tok@w}{\def\PY@tc##1{\textcolor[rgb]{0.73,0.73,0.73}{##1}}}
\@namedef{PY@tok@c}{\let\PY@it=\textit\def\PY@tc##1{\textcolor[rgb]{0.24,0.48,0.48}{##1}}}
\@namedef{PY@tok@cp}{\def\PY@tc##1{\textcolor[rgb]{0.61,0.40,0.00}{##1}}}
\@namedef{PY@tok@k}{\let\PY@bf=\textbf\def\PY@tc##1{\textcolor[rgb]{0.00,0.50,0.00}{##1}}}
\@namedef{PY@tok@kp}{\def\PY@tc##1{\textcolor[rgb]{0.00,0.50,0.00}{##1}}}
\@namedef{PY@tok@kt}{\def\PY@tc##1{\textcolor[rgb]{0.69,0.00,0.25}{##1}}}
\@namedef{PY@tok@o}{\def\PY@tc##1{\textcolor[rgb]{0.40,0.40,0.40}{##1}}}
\@namedef{PY@tok@ow}{\let\PY@bf=\textbf\def\PY@tc##1{\textcolor[rgb]{0.67,0.13,1.00}{##1}}}
\@namedef{PY@tok@nb}{\def\PY@tc##1{\textcolor[rgb]{0.00,0.50,0.00}{##1}}}
\@namedef{PY@tok@nf}{\def\PY@tc##1{\textcolor[rgb]{0.00,0.00,1.00}{##1}}}
\@namedef{PY@tok@nc}{\let\PY@bf=\textbf\def\PY@tc##1{\textcolor[rgb]{0.00,0.00,1.00}{##1}}}
\@namedef{PY@tok@nn}{\let\PY@bf=\textbf\def\PY@tc##1{\textcolor[rgb]{0.00,0.00,1.00}{##1}}}
\@namedef{PY@tok@ne}{\let\PY@bf=\textbf\def\PY@tc##1{\textcolor[rgb]{0.80,0.25,0.22}{##1}}}
\@namedef{PY@tok@nv}{\def\PY@tc##1{\textcolor[rgb]{0.10,0.09,0.49}{##1}}}
\@namedef{PY@tok@no}{\def\PY@tc##1{\textcolor[rgb]{0.53,0.00,0.00}{##1}}}
\@namedef{PY@tok@nl}{\def\PY@tc##1{\textcolor[rgb]{0.46,0.46,0.00}{##1}}}
\@namedef{PY@tok@ni}{\let\PY@bf=\textbf\def\PY@tc##1{\textcolor[rgb]{0.44,0.44,0.44}{##1}}}
\@namedef{PY@tok@na}{\def\PY@tc##1{\textcolor[rgb]{0.41,0.47,0.13}{##1}}}
\@namedef{PY@tok@nt}{\let\PY@bf=\textbf\def\PY@tc##1{\textcolor[rgb]{0.00,0.50,0.00}{##1}}}
\@namedef{PY@tok@nd}{\def\PY@tc##1{\textcolor[rgb]{0.67,0.13,1.00}{##1}}}
\@namedef{PY@tok@s}{\def\PY@tc##1{\textcolor[rgb]{0.73,0.13,0.13}{##1}}}
\@namedef{PY@tok@sd}{\let\PY@it=\textit\def\PY@tc##1{\textcolor[rgb]{0.73,0.13,0.13}{##1}}}
\@namedef{PY@tok@si}{\let\PY@bf=\textbf\def\PY@tc##1{\textcolor[rgb]{0.64,0.35,0.47}{##1}}}
\@namedef{PY@tok@se}{\let\PY@bf=\textbf\def\PY@tc##1{\textcolor[rgb]{0.67,0.36,0.12}{##1}}}
\@namedef{PY@tok@sr}{\def\PY@tc##1{\textcolor[rgb]{0.64,0.35,0.47}{##1}}}
\@namedef{PY@tok@ss}{\def\PY@tc##1{\textcolor[rgb]{0.10,0.09,0.49}{##1}}}
\@namedef{PY@tok@sx}{\def\PY@tc##1{\textcolor[rgb]{0.00,0.50,0.00}{##1}}}
\@namedef{PY@tok@m}{\def\PY@tc##1{\textcolor[rgb]{0.40,0.40,0.40}{##1}}}
\@namedef{PY@tok@gh}{\let\PY@bf=\textbf\def\PY@tc##1{\textcolor[rgb]{0.00,0.00,0.50}{##1}}}
\@namedef{PY@tok@gu}{\let\PY@bf=\textbf\def\PY@tc##1{\textcolor[rgb]{0.50,0.00,0.50}{##1}}}
\@namedef{PY@tok@gd}{\def\PY@tc##1{\textcolor[rgb]{0.63,0.00,0.00}{##1}}}
\@namedef{PY@tok@gi}{\def\PY@tc##1{\textcolor[rgb]{0.00,0.52,0.00}{##1}}}
\@namedef{PY@tok@gr}{\def\PY@tc##1{\textcolor[rgb]{0.89,0.00,0.00}{##1}}}
\@namedef{PY@tok@ge}{\let\PY@it=\textit}
\@namedef{PY@tok@gs}{\let\PY@bf=\textbf}
\@namedef{PY@tok@gp}{\let\PY@bf=\textbf\def\PY@tc##1{\textcolor[rgb]{0.00,0.00,0.50}{##1}}}
\@namedef{PY@tok@go}{\def\PY@tc##1{\textcolor[rgb]{0.44,0.44,0.44}{##1}}}
\@namedef{PY@tok@gt}{\def\PY@tc##1{\textcolor[rgb]{0.00,0.27,0.87}{##1}}}
\@namedef{PY@tok@err}{\def\PY@bc##1{{\setlength{\fboxsep}{\string -\fboxrule}\fcolorbox[rgb]{1.00,0.00,0.00}{1,1,1}{\strut ##1}}}}
\@namedef{PY@tok@kc}{\let\PY@bf=\textbf\def\PY@tc##1{\textcolor[rgb]{0.00,0.50,0.00}{##1}}}
\@namedef{PY@tok@kd}{\let\PY@bf=\textbf\def\PY@tc##1{\textcolor[rgb]{0.00,0.50,0.00}{##1}}}
\@namedef{PY@tok@kn}{\let\PY@bf=\textbf\def\PY@tc##1{\textcolor[rgb]{0.00,0.50,0.00}{##1}}}
\@namedef{PY@tok@kr}{\let\PY@bf=\textbf\def\PY@tc##1{\textcolor[rgb]{0.00,0.50,0.00}{##1}}}
\@namedef{PY@tok@bp}{\def\PY@tc##1{\textcolor[rgb]{0.00,0.50,0.00}{##1}}}
\@namedef{PY@tok@fm}{\def\PY@tc##1{\textcolor[rgb]{0.00,0.00,1.00}{##1}}}
\@namedef{PY@tok@vc}{\def\PY@tc##1{\textcolor[rgb]{0.10,0.09,0.49}{##1}}}
\@namedef{PY@tok@vg}{\def\PY@tc##1{\textcolor[rgb]{0.10,0.09,0.49}{##1}}}
\@namedef{PY@tok@vi}{\def\PY@tc##1{\textcolor[rgb]{0.10,0.09,0.49}{##1}}}
\@namedef{PY@tok@vm}{\def\PY@tc##1{\textcolor[rgb]{0.10,0.09,0.49}{##1}}}
\@namedef{PY@tok@sa}{\def\PY@tc##1{\textcolor[rgb]{0.73,0.13,0.13}{##1}}}
\@namedef{PY@tok@sb}{\def\PY@tc##1{\textcolor[rgb]{0.73,0.13,0.13}{##1}}}
\@namedef{PY@tok@sc}{\def\PY@tc##1{\textcolor[rgb]{0.73,0.13,0.13}{##1}}}
\@namedef{PY@tok@dl}{\def\PY@tc##1{\textcolor[rgb]{0.73,0.13,0.13}{##1}}}
\@namedef{PY@tok@s2}{\def\PY@tc##1{\textcolor[rgb]{0.73,0.13,0.13}{##1}}}
\@namedef{PY@tok@sh}{\def\PY@tc##1{\textcolor[rgb]{0.73,0.13,0.13}{##1}}}
\@namedef{PY@tok@s1}{\def\PY@tc##1{\textcolor[rgb]{0.73,0.13,0.13}{##1}}}
\@namedef{PY@tok@mb}{\def\PY@tc##1{\textcolor[rgb]{0.40,0.40,0.40}{##1}}}
\@namedef{PY@tok@mf}{\def\PY@tc##1{\textcolor[rgb]{0.40,0.40,0.40}{##1}}}
\@namedef{PY@tok@mh}{\def\PY@tc##1{\textcolor[rgb]{0.40,0.40,0.40}{##1}}}
\@namedef{PY@tok@mi}{\def\PY@tc##1{\textcolor[rgb]{0.40,0.40,0.40}{##1}}}
\@namedef{PY@tok@il}{\def\PY@tc##1{\textcolor[rgb]{0.40,0.40,0.40}{##1}}}
\@namedef{PY@tok@mo}{\def\PY@tc##1{\textcolor[rgb]{0.40,0.40,0.40}{##1}}}
\@namedef{PY@tok@ch}{\let\PY@it=\textit\def\PY@tc##1{\textcolor[rgb]{0.24,0.48,0.48}{##1}}}
\@namedef{PY@tok@cm}{\let\PY@it=\textit\def\PY@tc##1{\textcolor[rgb]{0.24,0.48,0.48}{##1}}}
\@namedef{PY@tok@cpf}{\let\PY@it=\textit\def\PY@tc##1{\textcolor[rgb]{0.24,0.48,0.48}{##1}}}
\@namedef{PY@tok@c1}{\let\PY@it=\textit\def\PY@tc##1{\textcolor[rgb]{0.24,0.48,0.48}{##1}}}
\@namedef{PY@tok@cs}{\let\PY@it=\textit\def\PY@tc##1{\textcolor[rgb]{0.24,0.48,0.48}{##1}}}

\def\PYZbs{\char`\\}
\def\PYZus{\char`\_}
\def\PYZob{\char`\{}
\def\PYZcb{\char`\}}
\def\PYZca{\char`\^}
\def\PYZam{\char`\&}
\def\PYZlt{\char`\<}
\def\PYZgt{\char`\>}
\def\PYZsh{\char`\#}
\def\PYZpc{\char`\%}
\def\PYZdl{\char`\$}
\def\PYZhy{\char`\-}
\def\PYZsq{\char`\'}
\def\PYZdq{\char`\"}
\def\PYZti{\char`\~}
% for compatibility with earlier versions
\def\PYZat{@}
\def\PYZlb{[}
\def\PYZrb{]}
\makeatother


    % For linebreaks inside Verbatim environment from package fancyvrb.
    \makeatletter
        \newbox\Wrappedcontinuationbox
        \newbox\Wrappedvisiblespacebox
        \newcommand*\Wrappedvisiblespace {\textcolor{red}{\textvisiblespace}}
        \newcommand*\Wrappedcontinuationsymbol {\textcolor{red}{\llap{\tiny$\m@th\hookrightarrow$}}}
        \newcommand*\Wrappedcontinuationindent {3ex }
        \newcommand*\Wrappedafterbreak {\kern\Wrappedcontinuationindent\copy\Wrappedcontinuationbox}
        % Take advantage of the already applied Pygments mark-up to insert
        % potential linebreaks for TeX processing.
        %        {, <, #, %, $, ' and ": go to next line.
        %        _, }, ^, &, >, - and ~: stay at end of broken line.
        % Use of \textquotesingle for straight quote.
        \newcommand*\Wrappedbreaksatspecials {%
            \def\PYGZus{\discretionary{\char`\_}{\Wrappedafterbreak}{\char`\_}}%
            \def\PYGZob{\discretionary{}{\Wrappedafterbreak\char`\{}{\char`\{}}%
            \def\PYGZcb{\discretionary{\char`\}}{\Wrappedafterbreak}{\char`\}}}%
            \def\PYGZca{\discretionary{\char`\^}{\Wrappedafterbreak}{\char`\^}}%
            \def\PYGZam{\discretionary{\char`\&}{\Wrappedafterbreak}{\char`\&}}%
            \def\PYGZlt{\discretionary{}{\Wrappedafterbreak\char`\<}{\char`\<}}%
            \def\PYGZgt{\discretionary{\char`\>}{\Wrappedafterbreak}{\char`\>}}%
            \def\PYGZsh{\discretionary{}{\Wrappedafterbreak\char`\#}{\char`\#}}%
            \def\PYGZpc{\discretionary{}{\Wrappedafterbreak\char`\%}{\char`\%}}%
            \def\PYGZdl{\discretionary{}{\Wrappedafterbreak\char`\$}{\char`\$}}%
            \def\PYGZhy{\discretionary{\char`\-}{\Wrappedafterbreak}{\char`\-}}%
            \def\PYGZsq{\discretionary{}{\Wrappedafterbreak\textquotesingle}{\textquotesingle}}%
            \def\PYGZdq{\discretionary{}{\Wrappedafterbreak\char`\"}{\char`\"}}%
            \def\PYGZti{\discretionary{\char`\~}{\Wrappedafterbreak}{\char`\~}}%
        }
        % Some characters . , ; ? ! / are not pygmentized.
        % This macro makes them "active" and they will insert potential linebreaks
        \newcommand*\Wrappedbreaksatpunct {%
            \lccode`\~`\.\lowercase{\def~}{\discretionary{\hbox{\char`\.}}{\Wrappedafterbreak}{\hbox{\char`\.}}}%
            \lccode`\~`\,\lowercase{\def~}{\discretionary{\hbox{\char`\,}}{\Wrappedafterbreak}{\hbox{\char`\,}}}%
            \lccode`\~`\;\lowercase{\def~}{\discretionary{\hbox{\char`\;}}{\Wrappedafterbreak}{\hbox{\char`\;}}}%
            \lccode`\~`\:\lowercase{\def~}{\discretionary{\hbox{\char`\:}}{\Wrappedafterbreak}{\hbox{\char`\:}}}%
            \lccode`\~`\?\lowercase{\def~}{\discretionary{\hbox{\char`\?}}{\Wrappedafterbreak}{\hbox{\char`\?}}}%
            \lccode`\~`\!\lowercase{\def~}{\discretionary{\hbox{\char`\!}}{\Wrappedafterbreak}{\hbox{\char`\!}}}%
            \lccode`\~`\/\lowercase{\def~}{\discretionary{\hbox{\char`\/}}{\Wrappedafterbreak}{\hbox{\char`\/}}}%
            \catcode`\.\active
            \catcode`\,\active
            \catcode`\;\active
            \catcode`\:\active
            \catcode`\?\active
            \catcode`\!\active
            \catcode`\/\active
            \lccode`\~`\~
        }
    \makeatother

    \let\OriginalVerbatim=\Verbatim
    \makeatletter
    \renewcommand{\Verbatim}[1][1]{%
        %\parskip\z@skip
        \sbox\Wrappedcontinuationbox {\Wrappedcontinuationsymbol}%
        \sbox\Wrappedvisiblespacebox {\FV@SetupFont\Wrappedvisiblespace}%
        \def\FancyVerbFormatLine ##1{\hsize\linewidth
            \vtop{\raggedright\hyphenpenalty\z@\exhyphenpenalty\z@
                \doublehyphendemerits\z@\finalhyphendemerits\z@
                \strut ##1\strut}%
        }%
        % If the linebreak is at a space, the latter will be displayed as visible
        % space at end of first line, and a continuation symbol starts next line.
        % Stretch/shrink are however usually zero for typewriter font.
        \def\FV@Space {%
            \nobreak\hskip\z@ plus\fontdimen3\font minus\fontdimen4\font
            \discretionary{\copy\Wrappedvisiblespacebox}{\Wrappedafterbreak}
            {\kern\fontdimen2\font}%
        }%

        % Allow breaks at special characters using \PYG... macros.
        \Wrappedbreaksatspecials
        % Breaks at punctuation characters . , ; ? ! and / need catcode=\active
        \OriginalVerbatim[#1,codes*=\Wrappedbreaksatpunct]%
    }
    \makeatother

    % Exact colors from NB
    \definecolor{incolor}{HTML}{303F9F}
    \definecolor{outcolor}{HTML}{D84315}
    \definecolor{cellborder}{HTML}{CFCFCF}
    \definecolor{cellbackground}{HTML}{F7F7F7}

    % prompt
    \makeatletter
    \newcommand{\boxspacing}{\kern\kvtcb@left@rule\kern\kvtcb@boxsep}
    \makeatother
    \newcommand{\prompt}[4]{
        {\ttfamily\llap{{\color{#2}[#3]:\hspace{3pt}#4}}\vspace{-\baselineskip}}
    }
    

    
    % Prevent overflowing lines due to hard-to-break entities
    \sloppy
    % Setup hyperref package
    \hypersetup{
      breaklinks=true,  % so long urls are correctly broken across lines
      colorlinks=true,
      urlcolor=urlcolor,
      linkcolor=linkcolor,
      citecolor=citecolor,
      }
    % Slightly bigger margins than the latex defaults
    
    \geometry{verbose,tmargin=1in,bmargin=1in,lmargin=1in,rmargin=1in}
    
    

\begin{document}
    
    \maketitle
    
    

    
    \textbf{NAME: Duo Zhou}

\textbf{NETID: duozhou2}

    \hypertarget{question-i}{%
\section{Question I}\label{question-i}}

\hypertarget{basic-question}{%
\subsection{Basic Question}\label{basic-question}}

Company ZZ wants to assign its staff members efficiently among the jobs
and wants to utilize their time. Based on people skills and job
requirements, we have a utility of \(u_{ij}\) for assigning person \(i\)
to job \(j\). The manager of ZZ wants to assign people to jobs to
maximize overall utility. How would you formulate this problem as a
network flow problem?

\hypertarget{solution}{%
\subsubsection{Solution}\label{solution}}

Introduce the variable \(x_{ij} \in \{0,1 \}^{n\times m}\), which
indicate:

\[x_{ij}=\begin{cases}1,\quad &Assign\ the\ i^{th}\ person\ to\ complete\ the\ j^{th}\ job\\0,\quad &Otherwise \end{cases}\]

Thus the mathematical model of the assignment problem is:

\[ \begin{align*} max\ & \sum_{i=1}^n \sum_{j=1}^m u_{ij} x_{ij} 
\\  s.t. & \sum_{j=1}^n x_{ij}= 1, \quad i=1,...,n
\\ & \sum_{i=1}^m x_{ij}= 1,  \quad j=1,...,m
\\ & x_{ij}\in \{0,1\} \end{align*}\]

    \hypertarget{bonus-question}{%
\subsection{Bonus Question}\label{bonus-question}}

Now assume that you have job assignments to be made for multiple time
periods. How would you change your formulation to meet this new
situation while still attempting to maximize the overall utility.

\hypertarget{solution}{%
\subsubsection{Solution}\label{solution}}

It can be assumed that each persion i could be assigned to each job j
with a time cost \(c_{ij}\), the total time of assigned jobs is at most
\(t_{i}\).

Introduce the variable \(x_{ij} \in \{0,1 \}^{n\times m}\) as well,
which indicate:

\[x_{ij}=\begin{cases}1,\quad &Assign\ the\ i^{th}\ person\ to\ complete\ the\ j^{th}\ job\\0,\quad &Otherwise \end{cases}\]

Thus the mathematical model of the assignment problem made for multiple
time periods is:

\[ \begin{align*} max\ &  \sum_{i=1}^n \sum_{j=1}^m u_{ij} x_{ij} 
\\ s.t. & \sum_{j=1}^m c_{ij} x_{ij}\leq t_i,  \quad i=1,...,n
\\   & \sum_{i=1}^n x_{ij}= 1, \quad j=1,...,m
\\ & x_{ij}\in \{0,1\} \end{align*}\]

    \hypertarget{question-ii}{%
\section{Question II}\label{question-ii}}

\hypertarget{basic-question}{%
\subsection{Basic Question}\label{basic-question}}

A global supply chain company has \(n\) factories (indexed by \(i\)) and
\(m\) warehouses (indexed by \(j\)). Each factory has a certain supply
available and each warehouse as a certain demand requirement. Assume
that the total supply is greater than the total demand for sake of
simplicity. The cost to ship one unit of item from factory \(i\) to
warehouse \(j\) is \(c_{ij}\). The supply chain manager wants to
minimize total shipment cost. Formulate the problem as a network flow
model.

\hypertarget{solution}{%
\subsubsection{Solution}\label{solution}}

Assumed that supply of factory \(i\) is \(s_i\), demand of warehouse
\(j\) is \(d_j\),

Introduce the variable \(x_{ij} \in N^{n\times m}\), which indicate the
quantity of products shiped from the \(i^{th}\) factory to the
\(j^{th}\) warehouse.

Thus the mathematical model of the transportation problem is:

\[ \begin{align*} min &  \sum_{i=1}^n \sum_{j=1}^m c_{ij} x_{ij} 
\\  s.t. & \sum_{j=1}^m x_{ij}\leq s_i,  \quad i=1,...,n
\\ & \sum_{i=1}^n x_{ij}\leq d_j, \quad j=1,...,m
\\ & x_{ij}\geq 0 \end{align*}\]

    \hypertarget{bonus}{%
\subsection{Bonus}\label{bonus}}

If there are multiple parts indexed by k and the unit shipping costs are
\(c_{ijk}\), and you have a unit size of \(d_{ijk}\), you want to limit
the shipments from each warehouse to a (say trucksize) \(D\). Revise
your above model.

\hypertarget{solution}{%
\subsubsection{Solution}\label{solution}}

Based on the above fomula, we can add a maximum shipment size constraint
states as the total size of the shipment(supply quantity \(\times\) item
size) from each warehouse should not exceed the maximum shipment size
\(D\) .

This can be expressed mathematically as:

\[ 
\begin{align*} min & \sum_{i=1}^{n}\sum_{j=1}^{m}\sum_{k=1}^{p} c_{ijk}x_{ijk} \\
s.t. &\sum_{j=1}^{m}\sum_{k=1}^{p} x_{ijk} \leq s_i, \quad  i=1,...,n\\
&\sum_{i=1}^{n}\sum_{k=1}^{p} x_{ijk} \leq d_j, \quad j=1,...,m\\
&\sum_{j=1}^{m}\sum_{k=1}^{p} d_{ijk}x_{ijk} \leq D, \quad  i=1,...,n\\
&x_{ijk} \geq 0  \end{align*}\]

    \hypertarget{question-iii}{%
\section{Question III}\label{question-iii}}

Paper and wood products companies need to define cutting schedules that
will maximize the total wood yield of their forests over some planning
period. Suppose that a company with control of \(p\) forest units wants
to identify the best cutting schedule over a planning horizon of \(k\)
years. Forest unit \(i\) has a total acreage of \(a_j\) units, and
studies that the company has undertaken predict that this unit will have
\(w_{ij}\) tons of woods available for harvesting in the year \(j\).
Based on its prediction of economic conditions, the company believes
that it should harvest at least \(I_j\) tons of wood in year \(j\). Due
to the availability of equipment and personnel, the company can harvest
at most \(U_j\) tons of wood in year \(j\). Formulate the problem of
determining a schedule with maximum wood yield as a network flow
problem.

\hypertarget{solution}{%
\subsubsection{Solution}\label{solution}}

Introduce the variable \(x_{ij} \in N^{n\times m}\), which indicate the
quantity in tons of woods cut from the \(i^{th}\) unit forest in the
\(j^{th}\) year.

So, the problem could be seemed as a IP problem where the decision
variables are \(x_{ij}\) and the constraints are linear. The objective
is to maximize the total wood yield by finding the optimal schedule for
harvesting wood from the forest units over the planning horizon of k
years, subject to the constraints of available wood, minimum and maximum
harvest, forest unit acreage and non-negativity as:

\[ \begin{align*} max &\sum_{i=1}^{p}\sum_{j=1}^{k} x_{ij} \\
s.t. &\sum_{j=1}^{k} x_{ij} \leq a_j, \quad  i = {1,2, \dots ,p}
\\ &\sum_{i=1}^{p} x_{ij} \geq I_j, \quad  j = {1,2, \dots ,k}
\\ &\sum_{i=1}^{p} x_{ij} \leq U_j, \quad  j = {1,2, \dots ,k}
\\ &x_{ij} \leq w_{ij}, \quad i = {1,2, \dots ,p},\quad j = {1,2, \dots ,k}
\\ &x_{ij} \geq 0  \end{align*}\]

    \hypertarget{question-iv}{%
\section{Question IV}\label{question-iv}}

For a network \(G = (V,E)\) with source node \(s\) and terminal node
\(t\), let \(c_{ij} \geq 0\) be the arc capacity for arc \((i,j)\in E\).

\hypertarget{write-down-the-lp-formulation-for-the-max-flow-problem.}{%
\subsection{Write down the LP formulation for the max-flow
problem.}\label{write-down-the-lp-formulation-for-the-max-flow-problem.}}

\hypertarget{solution}{%
\subsubsection{Solution}\label{solution}}

Set a source node \(s\in V\), a sink node \(t\in V\)

Introduce the variable \(x_{ij}\) represents the flow on arc \((i,j)\)
in the network. Then we can build a standrad LP max-flow problem formula
as: \[ \begin{align*}
max &\sum_{j \in V} x_{ij}
\\ s.t. &\sum_{j \in V} x_{ij} = \sum_{j \in V} x_{ji} \quad \forall i \in V\setminus {s,t}
\\ &x_{ij} \leq c_{ij} \quad \forall (i,j) \in E
\\ &x_{ij} \geq 0 \quad \forall (i,j) \in E \end{align*}\]

\hypertarget{write-the-lp-dual-of-the-formulation.}{%
\subsection{Write the LP dual of the
formulation.}\label{write-the-lp-dual-of-the-formulation.}}

\hypertarget{solution-1}{%
\subsubsection{Solution}\label{solution-1}}

Introduce variables: \(y_{i,j}\) represents the potential of arc
\({(i,j)}\) in the network.

Objective function: Minimize the sum of the potentials of all nodes in
the cut set.

This can be expressed mathematically as:

\[ \begin{align*}
min &\sum_{(i,j) \in E} c_{i,j}y_{i,j}
\\ s.t. &\sum_{(i,j) \in E}y_{i,j}-\sum_{(j,i) \in E}y_{j,i} = 0, \quad \forall i \in V\setminus {s,t}
\\ &\sum{y_{ij}} \leq 1 \quad \forall i \in V\setminus {s}
\\ &y_{i,j} \geq 0 \quad \forall (i,j) \in E \end{align*}\]

\hypertarget{in-your-own-words-provide-an-intuition-behind-the-dual-problem.}{%
\subsection{In your own words, provide an intuition behind the dual
problem.}\label{in-your-own-words-provide-an-intuition-behind-the-dual-problem.}}

\hypertarget{solution-2}{%
\subsubsection{Solution}\label{solution-2}}

The intuition behind the LP dual of the max-flow problem is that it
provides a way to understand the capacity constraints of the network
from a different perspective. The primal LP formulation of the max-flow
problem is focused on finding the maximum flow that can be sent from the
source to the terminal node, subject to the capacity constraints of the
arcs. In contrast, the LP dual of the max-flow problem, also known as
the ``flow-interdiction LP'', is focused on finding the minimum amount
of flow that can be blocked in order to disconnect the source and the
terminal nodes.

The dual problem's decision variables, \(y_i\), represent the flow on
each node in the network, while the slack variables, \(z_{ij}\),
represent the amount by which the flow on each arc exceeds its capacity
constraint. The objective function of the dual problem is to minimize
the total flow from the source to the terminal node, which is equivalent
to finding the minimum cut of the network. In this way, the LP dual
problem provides an alternative way of understanding the capacity
constraints of the network.

    \hypertarget{question-v}{%
\section{Question V}\label{question-v}}

    The figure below shows an instance of the multi-commodity flow problem.
The network has 2 commodities, and a source and sink node for each
commodity. There are 6 transshipment nodes. The arc costs are given
alongside. In all but one arc (from node 2 to node 5), assume that the
capacity is infinity.

\hypertarget{formulate-the-multi-commodity-flow-problem-as-an-lp.}{%
\subsection{Formulate the multi-commodity flow problem as an
LP.}\label{formulate-the-multi-commodity-flow-problem-as-an-lp.}}

\hypertarget{solution}{%
\subsubsection{Solution}\label{solution}}

Introduce \(f(i,j)\) denotes the flow of arc \((i,j)\).

\[\begin{align*}
min &f_{12}+ 5f_{14} +f_{25} +f_{32}+6f_{36}+f_{54}+f_{56} \\
s.t. &f_{12}+ f_{14} =5 \\
&f_{32}+f_{36}=2 \\
&f_{14} +f_{54}=5 \\
&f_{36} +f_{56}=2 \\
&f_{12} +f_{32}=f_{25} \\
&f_{25} =f_{54}+f_{56}\\
&f_{25} \leq 5 \\
&f_{ij} \geq 0, \forall (i,j) \in E
\end{align*}\]

\hypertarget{find-the-optimal-solution-using-any-method-by-handa-computer-program.}{%
\subsection{Find the optimal solution (using any method -- by hand/a
computer
program).}\label{find-the-optimal-solution-using-any-method-by-handa-computer-program.}}

\hypertarget{solution-1}{%
\subsubsection{Solution}\label{solution-1}}

Solve by networkx

    \begin{tcolorbox}[breakable, size=fbox, boxrule=1pt, pad at break*=1mm,colback=cellbackground, colframe=cellborder]
\prompt{In}{incolor}{1}{\boxspacing}
\begin{Verbatim}[commandchars=\\\{\}]
\PY{k+kn}{import} \PY{n+nn}{numpy} \PY{k}{as} \PY{n+nn}{np}
\PY{k+kn}{import} \PY{n+nn}{matplotlib}\PY{n+nn}{.}\PY{n+nn}{pyplot} \PY{k}{as} \PY{n+nn}{plt}
\PY{k+kn}{import} \PY{n+nn}{networkx} \PY{k}{as} \PY{n+nn}{nx}
\PY{n}{G1} \PY{o}{=} \PY{n}{nx}\PY{o}{.}\PY{n}{DiGraph}\PY{p}{(}\PY{p}{)} 
\PY{n}{G1}\PY{o}{.}\PY{n}{add\PYZus{}edges\PYZus{}from}\PY{p}{(}\PY{p}{[}\PY{p}{(}\PY{l+s+s1}{\PYZsq{}}\PY{l+s+s1}{v1}\PY{l+s+s1}{\PYZsq{}}\PY{p}{,}\PY{l+s+s1}{\PYZsq{}}\PY{l+s+s1}{v2}\PY{l+s+s1}{\PYZsq{}}\PY{p}{,}\PY{p}{\PYZob{}}\PY{l+s+s1}{\PYZsq{}}\PY{l+s+s1}{weight}\PY{l+s+s1}{\PYZsq{}}\PY{p}{:} \PY{l+m+mi}{1}\PY{p}{\PYZcb{}}\PY{p}{)}\PY{p}{,}
                  \PY{p}{(}\PY{l+s+s1}{\PYZsq{}}\PY{l+s+s1}{v1}\PY{l+s+s1}{\PYZsq{}}\PY{p}{,}\PY{l+s+s1}{\PYZsq{}}\PY{l+s+s1}{v4}\PY{l+s+s1}{\PYZsq{}}\PY{p}{,}\PY{p}{\PYZob{}}\PY{l+s+s1}{\PYZsq{}}\PY{l+s+s1}{weight}\PY{l+s+s1}{\PYZsq{}}\PY{p}{:} \PY{l+m+mi}{5}\PY{p}{\PYZcb{}}\PY{p}{)}\PY{p}{,}
                  \PY{p}{(}\PY{l+s+s1}{\PYZsq{}}\PY{l+s+s1}{v2}\PY{l+s+s1}{\PYZsq{}}\PY{p}{,}\PY{l+s+s1}{\PYZsq{}}\PY{l+s+s1}{v5}\PY{l+s+s1}{\PYZsq{}}\PY{p}{,}\PY{p}{\PYZob{}}\PY{l+s+s1}{\PYZsq{}}\PY{l+s+s1}{capacity}\PY{l+s+s1}{\PYZsq{}}\PY{p}{:} \PY{l+m+mi}{5}\PY{p}{,} \PY{l+s+s1}{\PYZsq{}}\PY{l+s+s1}{weight}\PY{l+s+s1}{\PYZsq{}}\PY{p}{:} \PY{l+m+mi}{1}\PY{p}{\PYZcb{}}\PY{p}{)}\PY{p}{,}
                  \PY{p}{(}\PY{l+s+s1}{\PYZsq{}}\PY{l+s+s1}{v3}\PY{l+s+s1}{\PYZsq{}}\PY{p}{,}\PY{l+s+s1}{\PYZsq{}}\PY{l+s+s1}{v2}\PY{l+s+s1}{\PYZsq{}}\PY{p}{,}\PY{p}{\PYZob{}}\PY{l+s+s1}{\PYZsq{}}\PY{l+s+s1}{weight}\PY{l+s+s1}{\PYZsq{}}\PY{p}{:} \PY{l+m+mi}{1}\PY{p}{\PYZcb{}}\PY{p}{)}\PY{p}{,}
                  \PY{p}{(}\PY{l+s+s1}{\PYZsq{}}\PY{l+s+s1}{v3}\PY{l+s+s1}{\PYZsq{}}\PY{p}{,}\PY{l+s+s1}{\PYZsq{}}\PY{l+s+s1}{v6}\PY{l+s+s1}{\PYZsq{}}\PY{p}{,}\PY{p}{\PYZob{}}\PY{l+s+s1}{\PYZsq{}}\PY{l+s+s1}{weight}\PY{l+s+s1}{\PYZsq{}}\PY{p}{:} \PY{l+m+mi}{6}\PY{p}{\PYZcb{}}\PY{p}{)}\PY{p}{,}
                  \PY{p}{(}\PY{l+s+s1}{\PYZsq{}}\PY{l+s+s1}{v5}\PY{l+s+s1}{\PYZsq{}}\PY{p}{,}\PY{l+s+s1}{\PYZsq{}}\PY{l+s+s1}{v4}\PY{l+s+s1}{\PYZsq{}}\PY{p}{,}\PY{p}{\PYZob{}}\PY{l+s+s1}{\PYZsq{}}\PY{l+s+s1}{weight}\PY{l+s+s1}{\PYZsq{}}\PY{p}{:} \PY{l+m+mi}{1}\PY{p}{\PYZcb{}}\PY{p}{)}\PY{p}{,}
                  \PY{p}{(}\PY{l+s+s1}{\PYZsq{}}\PY{l+s+s1}{v5}\PY{l+s+s1}{\PYZsq{}}\PY{p}{,}\PY{l+s+s1}{\PYZsq{}}\PY{l+s+s1}{v6}\PY{l+s+s1}{\PYZsq{}}\PY{p}{,}\PY{p}{\PYZob{}}\PY{l+s+s1}{\PYZsq{}}\PY{l+s+s1}{weight}\PY{l+s+s1}{\PYZsq{}}\PY{p}{:} \PY{l+m+mi}{1}\PY{p}{\PYZcb{}}\PY{p}{)}\PY{p}{]}\PY{p}{)}

\PY{c+c1}{\PYZsh{} nx.min\PYZus{}cost\PYZus{}flow()}
\PY{n}{G1}\PY{o}{.}\PY{n}{add\PYZus{}node}\PY{p}{(}\PY{l+s+s2}{\PYZdq{}}\PY{l+s+s2}{v1}\PY{l+s+s2}{\PYZdq{}}\PY{p}{,} \PY{n}{demand}\PY{o}{=}\PY{o}{\PYZhy{}}\PY{l+m+mi}{5}\PY{p}{)}  \PY{c+c1}{\PYZsh{} sources}
\PY{n}{G1}\PY{o}{.}\PY{n}{add\PYZus{}node}\PY{p}{(}\PY{l+s+s2}{\PYZdq{}}\PY{l+s+s2}{v3}\PY{l+s+s2}{\PYZdq{}}\PY{p}{,} \PY{n}{demand}\PY{o}{=}\PY{o}{\PYZhy{}}\PY{l+m+mi}{2}\PY{p}{)}  
\PY{n}{G1}\PY{o}{.}\PY{n}{add\PYZus{}node}\PY{p}{(}\PY{l+s+s2}{\PYZdq{}}\PY{l+s+s2}{v4}\PY{l+s+s2}{\PYZdq{}}\PY{p}{,} \PY{n}{demand}\PY{o}{=}\PY{l+m+mi}{5}\PY{p}{)}  \PY{c+c1}{\PYZsh{} sinks}
\PY{n}{G1}\PY{o}{.}\PY{n}{add\PYZus{}node}\PY{p}{(}\PY{l+s+s2}{\PYZdq{}}\PY{l+s+s2}{v6}\PY{l+s+s2}{\PYZdq{}}\PY{p}{,} \PY{n}{demand}\PY{o}{=}\PY{l+m+mi}{2}\PY{p}{)}

               
\end{Verbatim}
\end{tcolorbox}

    \begin{tcolorbox}[breakable, size=fbox, boxrule=1pt, pad at break*=1mm,colback=cellbackground, colframe=cellborder]
\prompt{In}{incolor}{2}{\boxspacing}
\begin{Verbatim}[commandchars=\\\{\}]
\PY{n}{pos}\PY{o}{=}\PY{p}{\PYZob{}}\PY{l+s+s1}{\PYZsq{}}\PY{l+s+s1}{v2}\PY{l+s+s1}{\PYZsq{}}\PY{p}{:}\PY{p}{(}\PY{l+m+mi}{0}\PY{p}{,}\PY{l+m+mi}{5}\PY{p}{)}\PY{p}{,}\PY{l+s+s1}{\PYZsq{}}\PY{l+s+s1}{v3}\PY{l+s+s1}{\PYZsq{}}\PY{p}{:}\PY{p}{(}\PY{l+m+mi}{0}\PY{p}{,}\PY{l+m+mi}{2}\PY{p}{)}\PY{p}{,}\PY{l+s+s1}{\PYZsq{}}\PY{l+s+s1}{v1}\PY{l+s+s1}{\PYZsq{}}\PY{p}{:}\PY{p}{(}\PY{l+m+mi}{0}\PY{p}{,}\PY{l+m+mi}{8}\PY{p}{)}\PY{p}{,}\PY{l+s+s1}{\PYZsq{}}\PY{l+s+s1}{v6}\PY{l+s+s1}{\PYZsq{}}\PY{p}{:}\PY{p}{(}\PY{l+m+mi}{10}\PY{p}{,}\PY{l+m+mi}{2}\PY{p}{)}\PY{p}{,}\PY{l+s+s1}{\PYZsq{}}\PY{l+s+s1}{v4}\PY{l+s+s1}{\PYZsq{}}\PY{p}{:}\PY{p}{(}\PY{l+m+mi}{10}\PY{p}{,}\PY{l+m+mi}{8}\PY{p}{)}\PY{p}{,}\PY{l+s+s1}{\PYZsq{}}\PY{l+s+s1}{v5}\PY{l+s+s1}{\PYZsq{}}\PY{p}{:}\PY{p}{(}\PY{l+m+mi}{10}\PY{p}{,}\PY{l+m+mi}{5}\PY{p}{)}\PY{p}{\PYZcb{}}  \PY{c+c1}{\PYZsh{} position}
\PY{n}{fig}\PY{p}{,} \PY{n}{ax} \PY{o}{=} \PY{n}{plt}\PY{o}{.}\PY{n}{subplots}\PY{p}{(}\PY{n}{figsize}\PY{o}{=}\PY{p}{(}\PY{l+m+mi}{8}\PY{p}{,}\PY{l+m+mi}{6}\PY{p}{)}\PY{p}{)}
\PY{n}{ax}\PY{o}{.}\PY{n}{text}\PY{p}{(}\PY{l+m+mi}{6}\PY{p}{,}\PY{l+m+mf}{2.5}\PY{p}{,}\PY{l+s+s2}{\PYZdq{}}\PY{l+s+s2}{youcans\PYZhy{}xupt}\PY{l+s+s2}{\PYZdq{}}\PY{p}{,}\PY{n}{color}\PY{o}{=}\PY{l+s+s1}{\PYZsq{}}\PY{l+s+s1}{gainsboro}\PY{l+s+s1}{\PYZsq{}}\PY{p}{)}
\PY{n}{ax}\PY{o}{.}\PY{n}{set\PYZus{}title}\PY{p}{(}\PY{l+s+s2}{\PYZdq{}}\PY{l+s+s2}{Minimum Cost Maximum Flow with NetworkX}\PY{l+s+s2}{\PYZdq{}}\PY{p}{)}
\PY{n}{nx}\PY{o}{.}\PY{n}{draw}\PY{p}{(}\PY{n}{G1}\PY{p}{,}\PY{n}{pos}\PY{p}{,}\PY{n}{with\PYZus{}labels}\PY{o}{=}\PY{k+kc}{True}\PY{p}{,}\PY{n}{node\PYZus{}color}\PY{o}{=}\PY{l+s+s1}{\PYZsq{}}\PY{l+s+s1}{c}\PY{l+s+s1}{\PYZsq{}}\PY{p}{,}\PY{n}{node\PYZus{}size}\PY{o}{=}\PY{l+m+mi}{300}\PY{p}{,}\PY{n}{font\PYZus{}size}\PY{o}{=}\PY{l+m+mi}{10}\PY{p}{)} 
\PY{n}{plt}\PY{o}{.}\PY{n}{axis}\PY{p}{(}\PY{l+s+s1}{\PYZsq{}}\PY{l+s+s1}{on}\PY{l+s+s1}{\PYZsq{}}\PY{p}{)}
\PY{n}{plt}\PY{o}{.}\PY{n}{show}\PY{p}{(}\PY{p}{)}
\end{Verbatim}
\end{tcolorbox}

    \begin{center}
    \adjustimage{max size={0.9\linewidth}{0.9\paperheight}}{output_10_0.png}
    \end{center}
    { \hspace*{\fill} \\}
    
    \begin{tcolorbox}[breakable, size=fbox, boxrule=1pt, pad at break*=1mm,colback=cellbackground, colframe=cellborder]
\prompt{In}{incolor}{3}{\boxspacing}
\begin{Verbatim}[commandchars=\\\{\}]
\PY{c+c1}{\PYZsh{} Min Cost}
\PY{n}{minFlowCost} \PY{o}{=} \PY{n}{nx}\PY{o}{.}\PY{n}{min\PYZus{}cost\PYZus{}flow\PYZus{}cost}\PY{p}{(}\PY{n}{G1}\PY{p}{)} 
\PY{n}{minFlowDict} \PY{o}{=} \PY{n}{nx}\PY{o}{.}\PY{n}{min\PYZus{}cost\PYZus{}flow}\PY{p}{(}\PY{n}{G1}\PY{p}{)} 

\PY{c+c1}{\PYZsh{} Trasfer Data Stucture}
\PY{n}{edgeCapacity} \PY{o}{=} \PY{n}{nx}\PY{o}{.}\PY{n}{get\PYZus{}edge\PYZus{}attributes}\PY{p}{(}\PY{n}{G1}\PY{p}{,} \PY{l+s+s1}{\PYZsq{}}\PY{l+s+s1}{weight}\PY{l+s+s1}{\PYZsq{}}\PY{p}{)}
\PY{n}{edgeLabel} \PY{o}{=} \PY{p}{\PYZob{}}\PY{p}{\PYZcb{}} 
\PY{c+c1}{\PYZsh{} Sort Labels of Edge}
\PY{k}{for} \PY{n}{i} \PY{o+ow}{in} \PY{n}{edgeCapacity}\PY{o}{.}\PY{n}{keys}\PY{p}{(}\PY{p}{)}\PY{p}{:}  
    \PY{n}{edgeLabel}\PY{p}{[}\PY{n}{i}\PY{p}{]} \PY{o}{=} \PY{l+s+sa}{f}\PY{l+s+s1}{\PYZsq{}}\PY{l+s+s1}{w=}\PY{l+s+si}{\PYZob{}}\PY{n}{edgeCapacity}\PY{p}{[}\PY{n}{i}\PY{p}{]}\PY{l+s+si}{:}\PY{l+s+si}{\PYZcb{}}\PY{l+s+s1}{\PYZsq{}}
\PY{n}{edgeLists} \PY{o}{=} \PY{p}{[}\PY{p}{]}

\PY{k}{for} \PY{n}{i} \PY{o+ow}{in} \PY{n}{minFlowDict}\PY{o}{.}\PY{n}{keys}\PY{p}{(}\PY{p}{)}\PY{p}{:}
    \PY{k}{for} \PY{n}{j} \PY{o+ow}{in} \PY{n}{minFlowDict}\PY{p}{[}\PY{n}{i}\PY{p}{]}\PY{o}{.}\PY{n}{keys}\PY{p}{(}\PY{p}{)}\PY{p}{:}
        \PY{n}{edgeLabel}\PY{p}{[}\PY{p}{(}\PY{n}{i}\PY{p}{,}\PY{n}{j}\PY{p}{)}\PY{p}{]} \PY{o}{+}\PY{o}{=} \PY{l+s+s1}{\PYZsq{}}\PY{l+s+s1}{,f=}\PY{l+s+s1}{\PYZsq{}} \PY{o}{+} \PY{n+nb}{str}\PY{p}{(}\PY{n}{minFlowDict}\PY{p}{[}\PY{n}{i}\PY{p}{]}\PY{p}{[}\PY{n}{j}\PY{p}{]}\PY{p}{)}
        \PY{k}{if} \PY{n}{minFlowDict}\PY{p}{[}\PY{n}{i}\PY{p}{]}\PY{p}{[}\PY{n}{j}\PY{p}{]} \PY{o}{\PYZgt{}} \PY{l+m+mi}{0}\PY{p}{:}
            \PY{n}{edgeLists}\PY{o}{.}\PY{n}{append}\PY{p}{(}\PY{p}{(}\PY{n}{i}\PY{p}{,} \PY{n}{j}\PY{p}{)}\PY{p}{)}

\PY{n+nb}{print}\PY{p}{(}\PY{l+s+s2}{\PYZdq{}}\PY{l+s+s2}{Minimum Cost:}\PY{l+s+si}{\PYZob{}\PYZcb{}}\PY{l+s+s2}{\PYZdq{}}\PY{o}{.}\PY{n}{format}\PY{p}{(}\PY{n}{minFlowCost}\PY{p}{)}\PY{p}{)}
\PY{n+nb}{print}\PY{p}{(}\PY{l+s+s2}{\PYZdq{}}\PY{l+s+s2}{Path and flow of the minimum cost flow:}\PY{l+s+s2}{\PYZdq{}}\PY{p}{,} \PY{n}{minFlowDict}\PY{p}{)} 
\PY{n+nb}{print}\PY{p}{(}\PY{l+s+s2}{\PYZdq{}}\PY{l+s+s2}{Path:}\PY{l+s+s2}{\PYZdq{}}\PY{p}{,} \PY{n}{edgeLists}\PY{p}{)} 
\end{Verbatim}
\end{tcolorbox}

    \begin{Verbatim}[commandchars=\\\{\}]
Minimum Cost:25
Path and flow of the minimum cost flow: \{'v1': \{'v2': 3, 'v4': 2\}, 'v2': \{'v5':
5\}, 'v4': \{\}, 'v5': \{'v4': 3, 'v6': 2\}, 'v3': \{'v2': 2, 'v6': 0\}, 'v6': \{\}\}
Path: [('v1', 'v2'), ('v1', 'v4'), ('v2', 'v5'), ('v5', 'v4'), ('v5', 'v6'),
('v3', 'v2')]
    \end{Verbatim}

    \begin{tcolorbox}[breakable, size=fbox, boxrule=1pt, pad at break*=1mm,colback=cellbackground, colframe=cellborder]
\prompt{In}{incolor}{4}{\boxspacing}
\begin{Verbatim}[commandchars=\\\{\}]
\PY{n}{fig}\PY{p}{,} \PY{n}{ax} \PY{o}{=} \PY{n}{plt}\PY{o}{.}\PY{n}{subplots}\PY{p}{(}\PY{n}{figsize}\PY{o}{=}\PY{p}{(}\PY{l+m+mi}{8}\PY{p}{,}\PY{l+m+mi}{6}\PY{p}{)}\PY{p}{)}
\PY{n}{ax}\PY{o}{.}\PY{n}{set\PYZus{}title}\PY{p}{(}\PY{l+s+s2}{\PYZdq{}}\PY{l+s+s2}{Capacity network with multi source and multi sink}\PY{l+s+s2}{\PYZdq{}}\PY{p}{)}
\PY{n}{nx}\PY{o}{.}\PY{n}{draw}\PY{p}{(}\PY{n}{G1}\PY{p}{,}\PY{n}{pos}\PY{p}{,}\PY{n}{with\PYZus{}labels}\PY{o}{=}\PY{k+kc}{True}\PY{p}{,}\PY{n}{node\PYZus{}color}\PY{o}{=}\PY{l+s+s1}{\PYZsq{}}\PY{l+s+s1}{skyblue}\PY{l+s+s1}{\PYZsq{}}\PY{p}{,}\PY{n}{node\PYZus{}size}\PY{o}{=}\PY{l+m+mi}{200}\PY{p}{,}\PY{n}{font\PYZus{}size}\PY{o}{=}\PY{l+m+mi}{10}\PY{p}{)}  
\PY{n}{edgeLabel1} \PY{o}{=} \PY{n}{nx}\PY{o}{.}\PY{n}{get\PYZus{}edge\PYZus{}attributes}\PY{p}{(}\PY{n}{G1}\PY{p}{,} \PY{l+s+s1}{\PYZsq{}}\PY{l+s+s1}{weight}\PY{l+s+s1}{\PYZsq{}}\PY{p}{)}
\PY{n}{nx}\PY{o}{.}\PY{n}{draw\PYZus{}networkx\PYZus{}nodes}\PY{p}{(}\PY{n}{G1}\PY{p}{,} \PY{n}{pos}\PY{p}{,} \PY{n}{nodelist}\PY{o}{=}\PY{p}{[}\PY{l+s+s1}{\PYZsq{}}\PY{l+s+s1}{v1}\PY{l+s+s1}{\PYZsq{}}\PY{p}{,}\PY{l+s+s1}{\PYZsq{}}\PY{l+s+s1}{v3}\PY{l+s+s1}{\PYZsq{}}\PY{p}{]}\PY{p}{,} \PY{n}{node\PYZus{}color}\PY{o}{=}\PY{l+s+s1}{\PYZsq{}}\PY{l+s+s1}{orange}\PY{l+s+s1}{\PYZsq{}}\PY{p}{)}  \PY{c+c1}{\PYZsh{}sources}
\PY{n}{nx}\PY{o}{.}\PY{n}{draw\PYZus{}networkx\PYZus{}nodes}\PY{p}{(}\PY{n}{G1}\PY{p}{,} \PY{n}{pos}\PY{p}{,} \PY{n}{nodelist}\PY{o}{=}\PY{p}{[}\PY{l+s+s1}{\PYZsq{}}\PY{l+s+s1}{v4}\PY{l+s+s1}{\PYZsq{}}\PY{p}{,}\PY{l+s+s1}{\PYZsq{}}\PY{l+s+s1}{v6}\PY{l+s+s1}{\PYZsq{}}\PY{p}{]}\PY{p}{,} \PY{n}{node\PYZus{}color}\PY{o}{=}\PY{l+s+s1}{\PYZsq{}}\PY{l+s+s1}{c}\PY{l+s+s1}{\PYZsq{}}\PY{p}{)}       \PY{c+c1}{\PYZsh{}sinks}
\PY{n}{nx}\PY{o}{.}\PY{n}{draw\PYZus{}networkx\PYZus{}edge\PYZus{}labels}\PY{p}{(}\PY{n}{G1}\PY{p}{,}\PY{n}{pos}\PY{p}{,}\PY{n}{edgeLabel}\PY{p}{,}\PY{n}{font\PYZus{}size}\PY{o}{=}\PY{l+m+mi}{20}\PY{p}{,}\PY{n}{label\PYZus{}pos}\PY{o}{=}\PY{l+m+mf}{0.5}\PY{p}{)}  
\PY{n}{nx}\PY{o}{.}\PY{n}{draw\PYZus{}networkx\PYZus{}edges}\PY{p}{(}\PY{n}{G1}\PY{p}{,}\PY{n}{pos}\PY{p}{,}\PY{n}{edgelist}\PY{o}{=}\PY{n}{edgeLists}\PY{p}{,}\PY{n}{edge\PYZus{}color}\PY{o}{=}\PY{l+s+s1}{\PYZsq{}}\PY{l+s+s1}{m}\PY{l+s+s1}{\PYZsq{}}\PY{p}{,}\PY{n}{width}\PY{o}{=}\PY{l+m+mi}{1}\PY{p}{)}  
\PY{n}{plt}\PY{o}{.}\PY{n}{axis}\PY{p}{(}\PY{l+s+s1}{\PYZsq{}}\PY{l+s+s1}{on}\PY{l+s+s1}{\PYZsq{}}\PY{p}{)}
\PY{n}{plt}\PY{o}{.}\PY{n}{show}\PY{p}{(}\PY{p}{)}
\end{Verbatim}
\end{tcolorbox}

    \begin{center}
    \adjustimage{max size={0.9\linewidth}{0.9\paperheight}}{output_12_0.png}
    \end{center}
    { \hspace*{\fill} \\}
    

    % Add a bibliography block to the postdoc
    
    
    
\end{document}
